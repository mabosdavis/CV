%%%%%%%%%%%%%%%%%%%%%%%%%%%%%%%%%%%%%%%%%
% This document is based on a template available at
% http://www.LaTeXTemplates.com
%
% Original author:
% Trey Hunner (http://www.treyhunner.com/)
%%%%%%%%%%%%%%%%%%%%%%%%%%%%%%%%%%%%%%%%%

%----------------------------------------------------------------------------------------
%	PACKAGES AND OTHER DOCUMENT CONFIGURATIONS
%----------------------------------------------------------------------------------------

\documentclass{resume} % Use the custom resume.cls style

\usepackage[left=0.75in,top=0.6in,right=0.75in,bottom=0.6in]{geometry} % Document margins
\usepackage{fancyhdr}
\usepackage{lastpage}
\usepackage{amssymb}
\usepackage{url}

%\input{officialheader.tex} %If you want this file to compile to my CV, comment out this line and uncomment the next three
\name{Matthew C. Davis} 
\address{ \url{linkedin.com/in/matthew--davis} } 
\address{(928)-830-3107 \\ mabosdavis@gmail.com \\ Provo, UT} 


\pagestyle{fancy}
\fancyhf{}
\renewcommand{\headrulewidth}{0pt}
\setlength{\parindent}{0pt}
\cfoot{page \thepage\ of \pageref{LastPage}}
\begin{document}

December 6, 2022\\\\\\

Dear Dr. Erhardt and Dr. Moeckel,\\

I am writing to express my interest in the doctoral candidate position at the University of Kentucky. I learned of this opportunity through Dr. Greg Macfarlane and he encouraged me to apply. With a deep interest in public transit and two years of research experience, I am a strong candidate for this position.\\

I am currently pursuing my MS in Civil Engineering at Brigham Young University (BYU) and will graduate in December 2023. During my undergraduate studies at BYU, I worked as a research assistant under a PhD candidate where we evaluated various methods to identify data anomalies in Automated Traffic Signal Performance Measure (ATSPM) data. We used R to run linear regressions, histogram distribution analysis, and moving average and standard deviation techniques to determine the method that gave the most flexibility while effectively identifying data anomalies. My work focused on determining the standard deviation and bin width parameters for the moving average and standard deviation method using a sensitivity analysis, writing and iteratively revising the literature review and methodology sections, and narrowing down the input ATSPM data that the team used based on data completeness and the presence of anomalies. \\

During our ATSPM research, I felt thrown into a topic that I had no pre-existing knowledge of. It was my first research experience and I felt very out of my league. Starting out was confusing and frustrating, but I chose to take the opportunity to build my skills. I learned to utilize databases like TRID and Elsevier and to link relevant papers in citations managers. I developed note taking workflows that enabled me to work through papers thoroughly and quickly. The development of this methodology has assisted me in the writing process of five literature reviews on topics such as longitudinal analyses, effectiveness of state cellphone bans for drivers, emissions of Bus Rapid Transit (BRT) compared to commuter rail, and intelligent transportation systems. \\

My thesis is focused on evaluating the impact of Variable Message Signs (VMS), Road Weather Information Systems (RWIS), and traffic cameras on Utah roadways through evidence-based analyses. The methodology is split into a road-data analysis measuring the effect of VMS messages as well as a survey distributed to employees at the Utah Department of Transportation (UDOT) measuring attitudes towards and usage of VMS, RWIS, and traffic cameras. So far, I have cleaned and processed over 2 million VMS messages and joined it with three years of crash data to identify messages instructing drivers to divert when crashes occur downstream. My proposed methodology is to conduct 15-20 site-specific before/after analyses of VMS messages displayed upstream of crashes measuring the change in diversion rate after message activation. This same methodology will be applied to messages activaed in Box Elder Canyon, UT during winter weather events to measure their effect of speed reductions. Lastly, the survey contains questions that will help determine how UDOT employees use these devices, how the devices are in their UDOT work, improvements that can be made to the system, and additional locations for future placement. This project will provide UDOT with valuable feedback that will assist in guiding the deployment for these devices across the state of Utah. \\

Working on two very different research projects taught me the value of quick learning and adaptability. In addition, half-way through our timeline of the ATSPM research, the scope of the project was changed because of unexpected issues with the data. This required me to become familiar with running statistical tests and methods through R in a short time that I was not comfortable with. I know that with my range of experience and familiarity with quick learning, I have the ability to pick up the knowledge to understand transportation demand modeling and to become proficient in a short period of time. \\

My research interests lie in improving and adapting transit as a strong transport alternative to cars. I see it as the future of transportation in dense communities and want to be working on the forefront of developing methods for increasing ridership and improving efficiency. I have always been interested in how systems work and as a child, would trace on maps with my finger all of the different paths from one location to another. I developed a passion for advocating for transit when reading Jarrett Walker’s {\em Human Transit}. His book opened my eyes to the science and art of designing efficient and appropriate transit systems for communities. In everyday life, I rely on transit to get where I need to go. My wife and I live on the corridor of our newly constructed BRT in Provo that connects me to BYU as well as to the local commuter rail, which I take to work at AECOM each week. My passion for improving transit brought me to AECOM’s world class transit team, where I work as an intern on the design of the University Line BRT in Houston, TX, which will be the United State’s longest BRT line at 25 miles long. All of these experiences have continuously solidified that this is where I want to focus my efforts. \\

While completing my Master’s thesis, I’ve really been drawn to a PhD program by the hard work and continual learning that are integral to the research process. Writing was always difficult for me, as it is for a lot of people, but I’ve found a lot of joy over the last year through developing routines and methods to create less friction in the process of producing great writing. I have a growth mindset about writing and see writing well as a lifelong task that I feel confident in tackling for the next 30 years through an academic career.\\

In terms of coding, I have learned Excel VBA, Python, and R over the last three years and have the capability and willingness to learn Java to be prepared for this position. \\\\


Respectfully, \\\\\\\\




Matthew Davis



\end{document}
