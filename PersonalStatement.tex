%%%%%%%%%%%%%%%%%%%%%%%%%%%%%%%%%%%%%%%%%
% This document is based on a template available at
% http://www.LaTeXTemplates.com
%
% Original author:
% Trey Hunner (http://www.treyhunner.com/)
%%%%%%%%%%%%%%%%%%%%%%%%%%%%%%%%%%%%%%%%%

%----------------------------------------------------------------------------------------
%	PACKAGES AND OTHER DOCUMENT CONFIGURATIONS
%----------------------------------------------------------------------------------------

\documentclass{resume} % Use the custom resume.cls style

\usepackage[left=0.75in,top=0.6in,right=0.75in,bottom=0.6in]{geometry} % Document margins
\usepackage{fancyhdr}
\usepackage{lastpage}
\usepackage{amssymb}
\usepackage{url}

%\input{officialheader.tex} %If you want this file to compile to my CV, comment out this line and uncomment the next three
\name{Matthew C. Davis} 
\address{ \url{linkedin.com/in/matthew--davis} } 
\address{(928)-830-3107 \\ mabosdavis@gmail.com \\ Provo, UT} 


\pagestyle{fancy}
\fancyhf{}
\renewcommand{\headrulewidth}{0pt}
\setlength{\parindent}{0pt}
\cfoot{page \thepage\ of \pageref{LastPage}}
\begin{document}

October 5, 2023\\\\\\

I am applying to pursue the doctoral degree of Civil Engineering at The University of Tennessee, Knoxville because of the unique research opportunities and mentorship it provides. I learned of this opportunity through Dr. Candace Brakewood and she encouraged me to apply. With a deep interest in public transit research and a desire to become a professor and teach, I know that this program is the best opportunity to help me achieve these goals. \\

Ever since I was young, I have been interested in how systems work. I have memories of tracing different paths on maps with my finger from one location to another. This desire to know, connect and understand how things worked continued through school as I came to enjoy math and science. I wanted a career that would allow me to use these skills, but at the same time, give me an opportunity to help people and serve the community. \\

During the Junior year of my bachelors degree, I found myself in an Urban Planning class. Our professor had us read Jarrett Walker’s {\em Human Transit} and we started having discussions about the benefits and design of public transportation systems. I started to see how intentional transit systems could bring communities together, reduce congestion, and improve quality of life. I started taking more classes that involved transit to learn as much as I could. I took Transportation Engineering, Air Quality, and Transportation Demand Modeling classes that all discussed more technical aspects of transit design. It felt like the missing puzzle piece that I had been looking for fell into place. By pursuing a career in transit, I could utilize my engineering skills while serving individuals and communities. It felt like the perfect combination of these two desires.\\

Due to this newfound direction, I applied for and started working as a transit intern at AECOM. In my role, I worked on designing some of the largest transit projects in the country, such as a new 25-mile Bus Rapid Transit corridor in Houston, TX. I touched on many aspects of the project, from calculating removal quantities to drafting alternate alignments. As I continued this work, I truly enjoyed the planning and design of transit projects, but started to see many of the challenges that transit faces as I had more exposure to the industry. Some of these challenges include recovering ridership from the COVID-19 pandemic, accurately forecasting future ridership, adjusting to changes in fare policy and collection, and ensuring equitable access. I realized that I wanted to do more to help the industry move forward and contribute to overcome even some of these challenges. \\

This has led me to see the great value that can come from continuing my education beyond my master’s degree to pursue a PhD. With a PhD, I can work in academia to research solutions to for the community and ways to better transit systems. In addition to research, I will be able to go on and teach and share my passion of transit with the next generation of students. This is the way I feel I can make the biggest impact in helping future engineers and policy makers see that the future is in transit. \\

To be best prepared to pursue this career path, I know that studying and being mentored by Dr. Candace Brakewood is the best preparation I can have to be successful in my goals. Dr. Brakewood is a leader in transit research and explores topics in areas I am interested in, such as the future of fare payment, national ridership trends, and the equity ramifications of advances in transit. While maintaining a strong research presence, she also remains humble, approachable, and available to mentor her students. She exemplifies the kind of professor and researcher that I aspire to be. \\ 

By pursuing my PhD at UTK, I would be able to get a great education, have the best mentor in Dr. Brakewood, and gain all the skills needed to pursue my dreams of going into academia. Through pursuing a PhD, I hope to continue growing my deep seeded passion for transit and do everything in my power to help communities and individuals through this transformative mode of transportation. \\



\end{document}
